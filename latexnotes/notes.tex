\documentclass[11pt]{amsart}
\usepackage{amsmath, amssymb, amsthm}
\usepackage{todonotes}

\DeclareMathOperator*{\argmax}{argmax}
\DeclareMathOperator*{\argmin}{argmin}

\theoremstyle{definition}
\newtheorem{definition}{Definition}
\newtheorem{theorem}{Theorem}
\newtheorem{proposition}{Proposition}
\newtheorem{corollary}{Corollary}
\newtheorem{problem}{Problem}
\theoremstyle{remark}
\newtheorem*{remark}{Remark}
\newtheorem*{example}{Example}
\newtheorem*{solution}{Solution}

\title{Prisoner's Dilemma}
\author{Arvid Lunnemark}

\begin{document}
\maketitle

\begin{definition}
  The \textit{prisoner's dilemma} is a symmetric two-player game with two actions, cooperate ($C$) and defect ($D$), where, if player 1 plays $a$ and player 2 plays $b$, player 1 gets payoff 
  \todo{phrase this in a nicer way}
  \begin{equation*}
    p(a,b) = \begin{bmatrix}
      R & S \\
      T & P
    \end{bmatrix}
  \end{equation*}
  if player 1 is the row player, and cooperate is the first action. We have $T > R > P > S$, and typically, we have the concrete values $T = 5$, $R = 3$, $P = 1$ and $S = 0$.
  \todo{(could also think about including the $2R > T + S$ condition here)}
\end{definition}

\begin{definition}
  A \textit{strategy} is a Moore machine (finite automaton with outputs) over the input and output alphabet $\{C, D\}$, with probability $1-p$ of following the correct transition and probability $p$ of following the incorrect transition. 
  \todo{(could also define it as probability $p$ of outputting the wrong character, but should be largely equivalent)}
  The strategy can be viewed as a function $s$ from a history string $h$ of the opponents moves to an action, i.e., $s : \{C, D\}^* \to \{C,D\}$.
\end{definition}

\begin{definition}
  The \textit{expected time-average payoff} of strategy $s_1$ when played against strategy $s_2$ is \begin{equation*}
    v(s_1, s_2) = \lim_{T \to \infty} E \left[ \frac{1}{T} \sum_{t = 1}^T p(s_1(h_{t-1, 2}), s_2(h_{t-1, 1})) \right]
  \end{equation*}
  where the history is defined as \begin{equation*}
    \begin{split}
    h_{t, 1} = h_{t-1, 1} \circ s_1(h_{t-1, 2})  \\
    h_{t, 2} = h_{t-1, 2} \circ s_2(h_{t-1, 1}) 
    \end{split}
  \end{equation*}
  \todo{(TODO redo this. the actual definition i'm using is the distribution in the stationary state of the markov chain as explained below)}
\end{definition}

\begin{definition}
  A \textit{population} of strategies $P = (S, f)$ is a set $S$ of strategies and a function $f : S \to (0,1]$ such that $\sum_{s \in S} f(s) = 1$, representing the frequency of each strategy in the population.
\end{definition}

\begin{definition}
  The \textit{fitness} of a strategy $s$ in a population $P = (S, f)$ is \begin{equation*}
    F(s) = \sum_{s' \in S} f(s) v(s, s').
  \end{equation*}
\end{definition}

\begin{definition}
  A strategy $s_2$ can \textit{invade} a strategy $s_1$, if there exists a real number $\alpha \in (0,1]$, such that for all populations $P$ with $S = \{s_1, s_2\}$ and $f(s_2) < \alpha$, we have
  \begin{equation*}
    F(s_2) > F(s_1).
  \end{equation*}
  \todo{a question is what to do with evolutionary drift, that is, if $F(s_2) = F(s_1)$, becuase it could potentially pave the way for more aggressive strategies. NOTE this is much more complicated than I first thought. the way I defined it now says that a strategy is evolutionary stable even if it can be taken over 99\% by another strategy, which feels weird. on the other hand, the way I defined it now implies that an evolutionary stable strategy will always stick around, even in minority, but it probably could easily be replaced after being in the minority. on the other hand, the prevailing definition also does not preclude the possibility of being easily replaced after evolutionary drift has taken place, and one should need to defend against that too if one wants the ultimate evolutionary super stable strategy. I feel like super stable is impossible though because if you were drifted with allC and then invaded by allD allD would take over and the only way to protect against it would be to go allD on allC but that would definitely not work I think. }
\end{definition}
\begin{remark}
  Thus, $s_2$ can invade $s_1$ if and only if it can start as an infinitesimally small part of the population and grow to become a constant fraction $\alpha$ of it.
\end{remark}

\begin{definition}
  A strategy $s_1$ is \textit{evolutionary stable} if it cannot be invaded by any strategy $s_2$.
\end{definition}

\begin{corollary}
  A strategy $s_1$ is evolutionary stable if and only if \begin{equation*}
    v(s_1,s_1) > v(s_2,s_1)
  \end{equation*}
  or \begin{equation*}
    v(s_1,s_1) = v(s_2, s_1) \text{ and } v(s_1,s_2) \geq v(s_2,s_2)
  \end{equation*}
  for all $s_2$.
  \todo{apparently the last inequality is strict in the traditional definition. given what I've seen with Pavlov it may even be possible to have that stricter definition be in place}
\end{corollary}
\begin{proof}
  Yeah this is true, the proof is easy by examination of cases. I just don't want to write it now.
\end{proof}

\begin{theorem}
  Suppose a strategy $s$ is evolutionary stable. Then $v(s,s) = R$. In other words, $s$ is efficient with itself.
  \todo{This looks like a very similar result to the theorem presented in the original paper. In fact, it looks like merely a simplication of it. To the contrary, however, it is very different, which stems from the definition of $v(\cdot,\cdot)$ — it captures the entire behavior, the actual expected value, instead of some weird artificial value of pretending that no mistakes occur at all which doesn't really make sense. Therefore, I think that this definition of evolutionary stable and $v$ will even exclude TFT from being evolutionary stable, which is a highly interesting result actually. Not sure if it is true though because that would mean that there would be an invader for TFT but I think that that is definitely a possibility, and that this model is actually more similar to real life than all other models presented in the literature on this subject so far. nice.}
\end{theorem}

OK we need a different notion of what it means to be efficient with oneself now. \textit{Probably.} In fact, maybe we don't. Because AllC will get $R$. But neither TFT nor Pavlov will. And, in fact, any strategy that is not stupid, will have opposite behavior when $p = 0$ and when $p = 1$ which suggests that it cannnot possibly get $R$ in both situations. COULD MAKE IT ONLY WORK IN THE LIMIT AS $p \to 0$.

wait..... maybe allC can invade TFT now. seems to depend on the specific values of $R, S, T, P$ which is the worst possible situation one can end up in. I think AllC can invade Pavlov too, which is not very promising

hmm okay so Pavlov is better against itself than TFT against itself, which, however, we have seen doesn't matter much.


I think the proof strategy is like this: suppose $v(s_1,s_1) < R$. then create a strategy $s_2$ that is more or less exactly like $s_1$, so $v(s_2,s_1) = v(s_1,s_2) = v(s_1,s_1)$, but $s_2$ identifies itself and has $v(s_2,s_2) = R$.

I think that captures the idea of an outside species pretending to be the same as everyone else but secretly getting a lot of value from cooperating with other members of the infiltrators, which I think makes a lot of sense from the anthropomorphic viewpoint.

A reasonable condition is \begin{equation*}
  \argmax_v \left[ (R, T, S, P) \cdot v \right] = (1, 0, 0, 0).
\end{equation*}
I think this is needed and crucial. Maybe one should even require that it's the unique maximum but I'm not exactly sure. This also seems like a very reasonable thing to assume, and is a generalization of the commonly used $R \geq (S + T)/2$. I like this. omg wait this condition makes no sense at all.

If we require unique maximum, then TFT can be invaded by allC, which kinda makes sense.

NICEEEE: As $p$ tends to 0, Pavlov tends to characteristic vector $(1,0,0,0)$ (i.e. payoff $R$) so I actually think that this theorem might be true (especially considering that AllC fares poorly against Pavlov). Since TFT have characteristic vector $(1/4,1/4,1/4,1/4)$, and would thus not be evolutionary stable here, this theorem indicates a stronger result than the theorem that it is based on.

omg wait the condition I proposed makes no sense at all hmmmm

oh wow read https://search.proquest.com/docview/235762955?accountid=12492 they came up with exactly the same Markov chain as I did and seems to have done more analysis on it. they seem to not do much with their results at all, however.

\end{document}